% How useable is the implementation for the user and how to use it anyways
\section{Discussion}

The Human Presenter Generator represents a sophisticated pipeline for creating realistic
talking-head animations, combining \gls{Zonos} for voice synthesis and \gls{memo} for facial
animation. While technically impressive, the system's practical usability presents both
opportunities and significant challenges for end users.

From a usability perspective, the application provides an intuitive two-step workflow through its
Vue.js-based web interface. Users begin by either recording voice samples directly through their
microphone or uploading existing audio files to the \gls{Zonos} component. The interface guides
users through entering the desired text and language parameters before generating synthetic speech.
Subsequently, the \gls{memo} component allows users to capture face images via webcam or upload
static photographs, which are then animated to match the generated audio. This streamlined approach
makes the complex underlying technology accessible to non-technical users through clear visual
feedback and straightforward controls.

However, the system's hardware requirements pose substantial barriers to widespread adoption. The
implementation demands significant computational resources, particularly GPU memory, with the
documentation indicating that even systems with 20GB of VRAM proved insufficient for running both
\gls{Zonos} and \gls{memo} simultaneously. The Docker configuration reveals that while the
\gls{memo} component requires NVIDIA GPU acceleration for acceptable performance, the \gls{Zonos}
service had to be configured to run without GPU acceleration due to memory constraints. This
hardware limitation forces users to either invest in high-end graphics hardware or accept
significantly degraded performance, making the system impractical for typical consumer hardware
configurations.

The application's architecture, built around separate FastAPI microservices for each AI component
and a Vue.js frontend, demonstrates good separation of concerns but introduces additional complexity
for deployment and maintenance. Users must manage multiple Docker containers and ensure proper GPU
    driver integration through the NVIDIA Container Toolkit. The reliance on local processing, while
    ensuring privacy and offline functionality, places the entire computational burden on the user's
    system rather than distributing it across cloud infrastructure.

Despite these hardware challenges, the system offers valuable advantages for users with appropriate
computational resources. The fully local processing ensures complete data privacy, as no voice
samples or images are transmitted to external services. The modular architecture allows for
customization and extension of individual components, making it suitable for research applications
or specialized use cases. The continuous workflow from voice synthesis to facial animation within a
single application eliminates the need for users to coordinate multiple separate tools and manual
file transfers between processing stages.
