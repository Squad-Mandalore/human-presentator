% How the actual implementation was done
\section{Results}

The implementation of the human-presentator project deviated significantly from the initially
planned monolithic approach described in the methodology section. During development, several
technical constraints and compatibility issues necessitated a fundamental architectural redesign
that resulted in a modular, service-oriented system.

The most significant change from the original methodology was the transition from a single-script
Python application to a distributed 'microservice' like architecture. This shift was primarily driven by
Python version incompatibilities between the core dependencies, specifically Zonos for voice
synthesis and Memo for face animation. Rather than attempting to reconcile these version conflicts
within a monolithic structure, the implementation adopted a containerized approach using Docker
Compose to isolate each service in its own environment with appropriate Python versions and
dependencies.

The final implementation consists of three primary services: a Vue.js frontend (presentor-vue)
serving as the user interface, a Zonos API service for voice recognition and synthesis running on
port 8001, and a Memo API service for face animation processing on port 8002. This modular
architecture allows each component to operate independently while maintaining clear communication
channels through RESTful APIs.

A critical implementation decision involved forking both the Zonos and Memo projects to adapt their
system APIs for the specific requirements of the human-presentator use case. The original Zonos and
memo implementation was modified to better integrate with the distributed architecture. The Memo
fork in particular was adapted to try optimize face animation processing but without real
achievements.

The containerization strategy employed Docker with NVIDIA GPU runtime support to leverage hardware
acceleration for the computationally intensive AI operations. The memo-api service specifically
utilizes NVIDIA CUDA capabilities for face animation processing, while the zonos-api service can
optionally leverage GPU acceleration when sufficient VRAM is available. As for now it is disabled
because it then would take up to 32GB VRAM.

\begin{figure}[H]
  \centering
  \includegraphics[width=\textwidth, keepaspectratio]{architecture.png}
  \caption{Architecture of the Human Presentor}
  \label{fig:architecture}
\end{figure}
