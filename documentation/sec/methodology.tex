% Architecture planning | How the application was planned to implement

\section{Methodology}

The human-presentator project was initially conceived as a streamlined, single-script Python
application designed to generate talking-head animations from static images and text input. The
original approach emphasized simplicity and cohesion, with all functionality consolidated into one
comprehensive Python script that would handle the entire pipeline from text input to final video
output.

Rather than developing proprietary solutions from scratch, the project methodology centered on
leveraging established open-source software for the core voice generation and image animation
components. This approach was chosen to reduce development time by building upon proven, well-tested
codebases, ensure compatibility with existing workflows and standards and maintain transparency and
auditability through open-source implementations.

An integral component of the original methodology included implementing automated validation
mechanisms for the generated MP4 output files. The automated validation approach aimed to ensure
consistent output quality while reducing manual review overhead during the development and testing
phases.

The initial architectural approach emphasized a monolithic design pattern, consolidating all
processing stages within a single executable unit. This methodology was selected to minimize
deployment complexity, reduce inter-component communication overhead, and provide a clear, linear
processing flow that could be easily understood and modified by developers working on the project.
